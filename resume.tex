% !TeX TS-program = lualatex

% Document class and font size
\documentclass[a4paper,8pt]{extarticle}

% Packages
\usepackage[utf8]{inputenc} % For input encoding
\usepackage{geometry} % For page margins
\usepackage{titlesec} % For section title formatting
\usepackage{enumitem} % For itemized list formatting
\usepackage[hidelinks]{hyperref} % For hyperlinks
\usepackage{url} % for styling hyperlinks
\usepackage{fontspec} % For custom fonts with xelatex and lualatex
\usepackage{titlesec} % Use titlesec to apply the new font to section headers

% Set the document margins
\geometry{a4paper, margin=0.75in} % Set paper size and margins

%-------------------------------------------------------------------------------
% Fonts
%-------------------------------------------------------------------------------

% How to define a font using a .ttc file (where all the fonts are in a single file):
% \newfontfamily{\Noteworthy}{Noteworthy.ttc}[
%     Ligatures = TeX ,
%     Path = /System/Library/Fonts/ ,
%     UprightFeatures = {FontIndex=0} ,
%     BoldFeatures = {FontIndex=1} ,
%     ItalicFeatures = {FontIndex=2} ,
%     BoldItalicFeatures = {FontIndex=3} ,
% ]

\newfontfamily{\TimesNewRoman}{Times New Roman}

% How to define a font using a .ttf file:
% Define a new font to be used for headers
\newfontfamily{\Garamond}{CormorantGaramond}[
    Ligatures = TeX ,
    Scale = 1.0,
    Path = /Library/Fonts/ ,
    Extension = .ttf ,
    UprightFont = *-regular ,
    BoldFont = *-bold ,
    ItalicFont = *-italic ,
    BoldItalicFont = *-bolditalic
]

% How to define the light version of a font in a .ttf file:
% % Define a new font to be used for headers
% \newfontfamily{\GaramondLight}{CormorantGaramond}[
%     Ligatures = TeX ,
%     Path = /Library/Fontsbtop/ ,
%     Extension = .ttf ,
%     UprightFont = *-Light ,
%     ItalicFont = *-LightItalic ,
% ]

% Define the light version of the helvetica font in the .ttc file
\newfontfamily\HelveticaLight{Helvetica.ttc}[
    Ligatures = TeX ,
    Path = /System/Library/Fonts/ ,
    UprightFeatures = {FontIndex=4} ,
]

% This command gives the ability to format text as the light font face, e.g. `\lightfont{Hello World}`
\newcommand\lightfont[1]{{\HelveticaLight#1}}

%-------------------------------------------------------------------------------
% Formatting
%-------------------------------------------------------------------------------

\begin{document}

% Set the main font
\setmainfont{Helvetica}

% Section header format
\titleformat{\section}
    {\Large\TimesNewRoman} % formatting and font
    {\thesection}{0em} % spacing between the section number and the title
    {} % the title
    [\titlerule] % additional code

% Section header spacing
\titlespacing*{\section}
    {0pt}
    {\baselineskip}
    {\baselineskip}

% Reduce the left margin of itemized bullet points
% and, remove separations between itemized bullet points
\setlist[itemize]
{
label=\textbullet,
itemindent=-7pt,
noitemsep}

% Disable page numbers
\pagestyle{empty}

% Format urls to use a certain font
\urlstyle{rm} % Use the default font for URLs

%-------------------------------------------------------------------------------
% Resume
%-------------------------------------------------------------------------------

% Retreive the macros from their respective files
%-------------------------------------------------------------------------------
% Summaries
%-------------------------------------------------------------------------------
\newcommand{\BartendingSummary}{I’m an experienced bartender with a passion for creating exceptional cocktails and providing excellent customer service. I have a proven track record of working efficiently in high-pressure environments while maintaining a friendly atmosphere and work environment. I am dedicated, reliable, and a team player looking to work with like-minded individuals who are committed to the success of the establishment.}

\newcommand{\ServingSummary}{I am an experienced service industry professional with a strong passion for providing exceptional customer service and creating memorable experiences for guests. With a proven ability to work efficiently in fast-paced environments, I excel in multitasking and maintaining a positive, welcoming atmosphere. I am dedicated, reliable, and a team player, committed to contributing to the success of the establishment.}

\newcommand{\FireSummary}{Self-motivated wildland firefighter with six months of experience as a U.S. Forest Service employee. Dedicated and physically fit individual looking to extend my passion for the outdoors and serving the public through a wildland firefighting position. I have a proven track record working in a team environment in high stress situations, and am eager to contribute my efforts towards wildfire prevention, supression, containment, and restoration efforts. Qualified FFT2 and EMT-Basic.}
%-------------------------------------------------------------------------------
% Education
%-------------------------------------------------------------------------------
\newcommand{\TAMU}{
\noindent
\textbf{Texas A\&M University}, College Station, TX \hfill December 2019\\
B.Sc. Double Major \hfill GPA: 3.12/4.0\\
Mathematics and Aerospace Engineering
}

\newcommand{\Lonestar}{
\noindent
\textbf{Lone Star College - CyFair}, Houston, TX \hfill May 2016\\
Associate of Science in Science \hfill GPA: 3.67/4.0
}
%-------------------------------------------------------------------------------
% Experience
%-------------------------------------------------------------------------------

\newcommand{\Sundowners}{
\noindent
\textbf{U.S. Forest Service, Crew 8 (Sundowners)} \hfill Ojai, CA\\
\textit{Forestry Technician - Wildland Firefighter} \hfill May 2024 - Present
\begin{itemize}
    \item Member of a 16-person Type II hand crew, adeptly navigating dynamic, high-stress, and hazardous environments
    \item Proficiently operated in steep and uneven terrain on extended shifts while carrying gear exceeding 45 lbs.
    \item Crew responsibilities included fire line construction, fuels suppression and reduction, and mop-up tasks
    \item Utilized hand tools to construct fire lines, ensuring effective containment and control
    \item Demonstrated leadership as the lead EMT, overseeing crew medical concerns and emergencies
    \item Diligently conducted daily vehicle inspections and station upkeep, ensuring operational readiness
    \item Maintained vigilant situational awareness, monitoring for potential hazards and changing conditions
    \item Communicated effectively with team members to achieve project goals while prioritizing team safety
\end{itemize}
}

\newcommand{\AstroCarto}{
\noindent
\textbf{AstroCarto} \hfill Santa Barbara, CA\\
\textit{Contracted Software Developer} \hfill December 2024 - March 2024
\begin{itemize}
    \item Designed and implemented a flask API that performs precise astronomy calculations given a moment in time
    \item Utilized JPL's DE441 ephemeris to accurately compute planetary state vectors
    \item Applied spherical geometry to calculate the boundary of the Earth's visible portion for a distant observer
    \item Utilized Lunar osculating orbital elements to calculate the position of the lunar ascending node
    \item Used the Skyfield library for advanced timekeeping, JPL ephemeris parsing, and planetary subpoint determination
    \item Successfully negotiated contract terms with clients, defining clear deliverables and specs to ensure project alignment
\end{itemize}
}

\newcommand{\OdysseySpaceResearch}{
\noindent
\textbf{Odyssey Space Research} \hfill Houston, TX\\
\textit{Aerospace Engineer and Software Developer} \hfill February 2020 - August 2022
\begin{itemize}
    \item Verified GNC/trajectory designs of ISS-visiting vehicles for rendezvous, proximity, and capture (RPOC) operations
    \item Determined solar array blockage based on vehicle position data using basic computational geometry and Python
    \item Worked on the National Team's Lunar Lander doing modeling and simulation development in C++
    \item Developed a planetary magnetic field model in C++ for an orbit determination/propagation simulation
    \item Developed code for a Node.js web application designed to display Orion spacecraft telemetry data
    \item Performed hardware and software integration for Orion and Lunar Lander Trick-based simulations
\end{itemize}
}

\newcommand{\NassauBayFD}{
\noindent
\textbf{Nassau Bay Fire Department} \hfill Nassau Bay, TX\\
\textit{Volunteer Firefighter/EMT} \hfill May 2020 - February 2022
\begin{itemize}
    \item Volunteered as a firefighter/EMT and provided protection of life and property to the Nassau Bay community
    \item Responded to 911 emergency service calls including fire, motor vehicle accidents, EMS calls, and rescue situations
    \item Provided initial medical care including patient assessment and stabilization until mutual aid EMS arrived on scene
    \item Actively participated in inspecting and maintaining all equipment around the station and on the fire apparatuses
    \item Completed weekly trainings such as confined space drills and practice in initial attacks and vertical ventilation
\end{itemize}
}

\newcommand{\SentriForce}{
\noindent
\textbf{SentriForce} \hfill Houston, TX\\
\textit{Contracted Full Stack Web Developer} \hfill June 2019 - August 2022
\begin{itemize}
    \item Designed and built a custom web application to toggle computers on remote networks
    \item Cut original task completion time by approximately 90\%
    \item Performed full-stack web development using Python/Django, HTML, CSS, andJavaScript
    \item Integrated a third-party REST API to fetch data from an online database
    \item Deployed web-application on an Ubuntu server using Nginx and Gunicorn
\end{itemize}
}

\newcommand{\MilkAndHoney}{
\noindent
\textbf{Milk \& Honey} \hfill Santa Barbara, CA\\
\textit{Bartender/Closing Shift Lead} \hfill June 2023 – March 2024
\begin{itemize}
    \item Entrusted with a key for opening and closing the restaurant
    \item Delivered full service at the bar top while providing exceptional customer service to patrons
    \item Cultivated relationships with regular customers, fostering a welcoming and friendly atmosphere
    \item Rigurously adhered to health and liquor regulations, ensuring a safe and enjoyable environment
    \item Actively contributed to the development of the cocktail menu, introducing drinks that boosted sales
    \item Demonstrated proficiency in multitasking, time-management, and maintaining cleanliness of the bar
    \item Ensured precision in drink preparation using a jigger and enhanced presentation with artistic garnishes
\end{itemize}
}

\newcommand{\Eureka}{
\noindent
\textbf{Eureka!} \hfill Santa Barbara, CA\\
\textit{Bartender} \hfill January 2023 – June 2023
\begin{itemize}
    \item Succesfully manager a 15-seat bar and efficiently prepared drinks for the entire restaurant
    \item Chosen as a primary night-time bartender, demonstrating reliability during high-demand hours
    \item Cultivated relationships with regular customers, fostering a welcoming and friendly atmosphere
    \item Managed bar inventory, ensuring that all supplies were well-stocked and prepared for busy shifts
    \item Crafted balanced cocktails using a jigger while providing outstanding customer service to patrons
    \item Collaborated effectively with team members to ensure seamless service and a positive experience for guests
\end{itemize}
}

\newcommand{\DoorDash}{
\noindent
\textbf{DoorDash} \hfill Houston, TX and Santa Barbara, CA\\
\textit{Delivery Driver} \hfill March 2020 – January 2023
\begin{itemize}
    \item Ensured accurate and timely deliveries through clear communication with customers
    \item Efficiently completed dispatched orders with a focus on punctuality and safe driving practices
    \item Maintained a 4+ star customer rating by communicating effectively and ensuring food is hot upon delivery
    \item Collaborated professionally with restaurant staff to secure and verify correct orders as per customer requests
\end{itemize}
}

\newcommand{\Marvinos}{
\noindent
\textbf{Marvino’s Italian Kitchen} \hfill Houston, TX\\
\textit{Barback} \hfill April 2019 – March 2020
\begin{itemize}
    \item Thoroughly cleaned the bar during closing shifts, including washing floor mats
    \item Maintained bar cleanliness by bussing customer seats and wiping down the bar top
    \item Greeted customers with silverware, bread, olive oil, and took initial non-alcoholic beverage orders
\end{itemize}
}

\newcommand{\LakeridgeTownhomes}{
\textbf{Lakeridge Townhomes} \hfill College Station, TX\\
\textit{Maintenance} \hfill August 2016 – December 2019
\begin{itemize}
    \item Ensured cleanliness of apartment grounds by removing visible trash
    \item Managed and completed maintenance requests for 150 apartment units
    \item Promoted from construction laborer to head of maintenance, demonstrating expertise and reliability
\end{itemize}
\textit{Construction Laborer} \hfill June 2016 – August 2016
\begin{itemize}
    \item Performed general labor for 10 intense weeks during construction of four new, 3-story apartment buildings
    \item Cleaned construction site including sweeping dirt, removing drywall and broken brick, moving doors, and picking up trash
    \item Utilized basic hand tools and heavy machinery, including skill saws, Bobcat loaders, and Genie forklifts
    \item Provided general maintenance for Lakeridge residents
\end{itemize}
}

\newcommand{\LakeridgeTownhomesMaintenance}{
\textbf{Lakeridge Townhomes} \hfill College Station, TX\\
\textit{Maintenance} \hfill August 2016 – December 2019
\begin{itemize}
    \item Ensured cleanliness of apartment grounds by removing visible trash
    \item Managed and completed maintenance requests for 150 apartment units
    \item Promoted from construction laborer to head of maintenance, demonstrating expertise and reliability
\end{itemize}
}

\newcommand{\LakeridgeTownhomesLaborer}{
\textbf{Lakeridge Townhomes} \hfill College Station, TX\\
\textit{Construction Laborer} \hfill June 2016 – August 2016
\begin{itemize}
    \item Performed general labor for 10 intense weeks during construction of four new, 3-story apartment buildings
    \item Cleaned construction site including sweeping dirt, removing drywall and broken brick, moving doors, and picking up trash
    \item Utilized basic hand tools and heavy machinery, including skill saws, Bobcat loaders, and Genie forklifts
    \item Provided general maintenance for Lakeridge residents
\end{itemize}
}

\newcommand{\Engineering}{\AstroCarto \OdysseySpaceResearch \SentriForce}
\newcommand{\FireEMS}{\Sundowners \NassauBayFD \LakeridgeTownhomes}
\newcommand{\ServiceIndustry}{\MilkAndHoney \Eureka \DoorDash \Marvinos}
%-------------------------------------------------------------------------------
% Projects
%-------------------------------------------------------------------------------
\newcommand{\Meatball}{
\noindent
\textbf{Meatball - Chess Engine (work in progress)} \hfill March 2022 - Present\\
\textit{Project Link:} \url{https://github.com/hgducharme/meatball}
\begin{itemize}
    \item Built a chess engine in C++17 entirely from scratch
    \item Board representation: utilizes 64-bit integers to encode piece locations (bitboards)
    \item Move generation: pre-computes and stores each piece's attack space for each square to increase speed during runtime
    \item Created a recursive perft function to debug move generation by comparing the move space to known online results
    \item Move search (WIP): utilizes Monte Carlo Tree Search (MCTS) to explore the search space and decide on the best move
    % \item Move evaluation: a neural network trained with self-play reinforcement-learning to evaulate the quality of a position
    \item Developed the code base with intense focus on readability and maintainability
    \item Maintain a comprehensive test suite of unit and integration tests with coverage report generation
    \item Designed an isolated build environment and continuous integration pipeline using make, Docker, and Github Actions
\end{itemize}
}

\newcommand{\InvertedPendulum}{
\noindent
\textbf{Inverted Pendulum Control System} \hfill July 2019 - September 2019\\
\textit{Project Link:} \url{https://github.com/hgducharme/inverted-pendulum}
\begin{itemize}
    \item Built an entirely custom inverted pendulum embedded system
    \item Derived the second-order nonlinear dynamic model using Lagrangian dynamics
    \item Converted the dynamic model to a linearized state-space model
    \item Built a simulation/validation environment in Python to help design the LQR controller
    \item Performed embedded software by implementing LQR control on a microcontroller using C++
\end{itemize}
}
%-------------------------------------------------------------------------------
% Skills, Coursework, and Certifications
%-------------------------------------------------------------------------------
\newcommand{\EngineeringSkills}{
\section*{SKILLS \& COURSEWORK}
\textbf{Programming Languages:} Python, C++, Node.js, HTML, CSS, JavaScript, Bash\\
\textbf{Math Coursework:} Real analysis, abstract algebra, differential geometry, complex analysis, partial differential equations, linear algebra I \& II, numerical methods, discrete mathematics, probability\\
\textbf{Engineering Coursework:} Spacecraft attitude dynamics and control (graduate level), vehicle stability and control, active controls, numerical methods, astrodynamics, computational aerospace structural design (graduate level)
}

\newcommand{\FFCertifications}{
\section*{CERTIFICATIONS}
\subsection*{Professional Qualifications}
\begin{itemize}
    \item EMT-Basic: California EMT and NREMT
    \item NFPA 1001: Structural Firefighter 1 and 2 — IFSAC Certified
    \item NFPA 1072: HAZMAT Awareness and Operations — IFSAC Certified
    \item NFPA 1051: L-180, S-110, S-130, S-190, S-212 — NWCG Certified
\end{itemize}
\subsection*{Red Card Qualifications}
\begin{itemize}
    \item FireFighter Type II (FFT2)
\end{itemize}
\subsection*{Additional Qualifications}
\begin{itemize}
    \item First Aid/CPR
    \item A-Faller/Beginner faller/Sawyer
    \item USFS Agency Motor Vehicle License; Standard 4x4
\end{itemize}
}

% Header
\begin{center}
{\Huge \TimesNewRoman \bfseries Hunter Ducharme}\\[5pt] % add vertical spacing
hgducharme@gmail.com $\cdot$ (281) 450-7154
\end{center}

\section*{EDUCATION}
\TAMU

\section*{EXPERIENCE}

\HighlightTools
\SundownersBrief
\AstroCarto
\OdysseySpaceResearch
\SentriForce

\section*{PROJECTS}
\Meatball
\InvertedPendulum

\EngineeringSkills

\end{document}